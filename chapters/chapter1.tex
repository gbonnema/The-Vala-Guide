\chapter{Some Background}
Vala is a new programming language that aims to bring modern programming language features to GNOME developers without imposing any additional runtime requirements and without using a different ABI compared to applications and libraries written in C.

valac, the Vala compiler, is a self-hosting compiler that translates Vala source code into C source and header files. It uses the GObject type system to create classes and interfaces declared in the Vala source code.

The syntax of Vala is similar to C\#, modified to better fit the GObject type system. Vala supports modern language features as the following:

\begin{itemize}
	\item Interfaces
	\item Properties
	\item Signals
	\item Foreach
	\item Lambda expressions
	\item Type inference for local variables
	\item Generics
	\item Non-null types
	\item Assisted memory management
	\item Exception handling
\end{itemize}

Vala is designed to allow access to existing C libraries, especially GObject-based libraries, without the need for runtime bindings. All that is needed to use a library with Vala is an API file, containing the class and method declarations in Vala syntax. Vala currently comes with bindings for GLib and GTK+. It's planned to provide generated bindings for the full GNOME Platform at a later stage.

Using classes and methods written in Vala from an application written in C is not difficult. The Vala library only has to install the generated header files and C applications may then access the GObject-based API of the Vala library as usual. It should also be easily possible to write a bindings generator for access to Vala libraries from applications written in e.g. C\# as the Vala parser is written as a library, so that all compile-time information is available when generating a binding.

More information about Vala is available at

	http://live.gnome.org/Vala/

\inputvalacodefile{src_org/main.vala}
